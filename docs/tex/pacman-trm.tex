\documentclass[12pt]{article}

\usepackage[dvips,letterpaper,margin=0.75in,bottom=0.75in]{geometry}
\usepackage{cite}
\usepackage{slashed}
\usepackage{graphicx}
\usepackage{amsmath}
\usepackage{capt-of}
\usepackage{xcolor}

\begin{document}

\title{Technical Reference Manual for the Pixel Array Control Monitor
  and Network (PACMAN) Card}
\author{Hans-Gerd Berns, Armin Karcher, Michael Mulhearn}

\maketitle

\section{Introduction}

The Pixel Array Control Monitor and Network (PACMAN) card is the warm
electronics (located outside of the cryostat) for the pixellated
liquid argon (LAr) time projection chamber (TPC) based on the LArPix
ASIC.  The PACMAN card provides power and digital I/O to arrays of
LArPix ASICs hosted on pixel tile cards located within the cryostat.
In the production system, each PACMAN card handles up to ten pixel
tile cards.  The PACMAN card provides analog test and monitor signals
as well as digital control signals for clock, exernal trigger, and
reset, and sync.  The PACMAN card serves as a timing system end point,
hosts a linux CPU, and is connected to the data acquisition system via
ethernet.

The PACMAN card plays a supporting role in the LArPix system.  It does
not face technical challenges as severe as the cold electronics of the
LArPix system and therefore a basic design constraint is that it in no
way limits the performance of the supported system.

\section{Prototype Results}

Many of the specifications for the PACMAN are driven by measurements,
using prototype devices, as described in this section.

The Intrinsic RMS noise on the LArPix ASIC was
benchtested\footnote{Reported 6 Mar 2020, Brooke} at $~\sim 3~\rm mV$.

The PACMAN card provides digital (VDDD) and analog (VDDA) power for
the LArPix ASICs.  The ASIC power draw during power up was
measured\footnote{Reported 13 Feb 2020, Brooke} as follows:
\captionof{table}{ASIC Current Draw at Power-up}
\begin{center}
\begin{tabular}{lllll}
   & Voltage (V) & Current (mA) & Power (mW) & Power ($\rm \mu W / chan$)\\
VDDA  & 1.8 & 1.4  & 2.5  & 39\\
VDDD  & 1.8 & 6.92 & 12.5 & 195\\
\end{tabular}
\end{center}

\newpage
The 10x10 pixel-tile power draw was measured \footnote{Reported 14 Jan 2021, Brooke} as follows:
\captionof{table}{10x10 Power Draw}
\begin{center}
\begin{tabular}{llll}
   & Voltage (V) & Current (mA) & Power (W) \\
VDDA  & 1.8 & 166  & 0.3  \\
VDDD  & 1.8 & 611  & 1.1  \\
\end{tabular}
\end{center}

\section{Specifications for the PACMAN Card}

\subsection{Overview}

Each PACMAN card provides power, digital I/O, and analog test and
monitoring signals to up to ten pixel tile cards.  Each pixel tile
card will consist of 160 ASICs.

\subsection{Power}

The PACMAN card provides digital and analog power to up to ten pixel
tile cards.  The digital voltage level (VDDD) and analog voltage level
(VDDA) are digitally controlled independently for each pixel tile
card, within the limits and at the stability specified below.

The power requirements of the PACMAN are based on scaling the measured
power consumption of the 100 ASIC pixel tile cards to 160 ASIC pixel
tile cards of the production system, a scale factor of 1.6.  An
additional (minimum) safety factor of $10\%$ is added to these
extrapolated power requirements to derive the following
specifications:
\captionof{table}{Digital and Analog Power Requirements (per tile card)}
\begin{center}
\begin{tabular}{llllll}
      & Stability & Min (V) & Max (V) & Max Current (A) & Max Power (W) \\
VDDA  & $1\%$ & 0 & 1.8 & 0.3 & 0.7\\
VDDD  & $1\%$ & 0 & 1.8 & 1.1 & 2.2\\
\end{tabular}
\end{center}

Reaching a $1\%$ stability on VDDA and VDDD requires linear regulation
and so the power requirements at the input of the PACMAN are derived
from a higher voltage level of $3~\rm V$.  This implies that the initial
switching voltage requirement must provide at least 42 W at $3~\rm V$.

The PACMAN power is provided by a single nominal 48~V 2~A DC input.
The power connection is the four position terminal connector (Phenoix
Contact P/N 5444660).  The inner two pins provide power and the outer
two are for sense:
\captionof{table}{Input Power Connections}
\begin{tabular}{ll}
pin & function \\
1 & V+ sense  \\
2 & V+ \\
3 & V- \\
4 & V- sense \\ 
\end{tabular}\\
The connector is rated for 8 A and 300 V.  The PACMAN is fused at 2.5
A and has over voltage protection at 60 V.  The sense pins are fused
at 0.5 A.

\subsection{Digital Signals}

The LArPix-v2 ASIC uses a bit-serial protocol to transmit and receive
data.  The PACMAN is the primary and the LARPix ASIC the secondary, so
that the PACMAN transmits POSI and receives PISO signals.

The PACMAN shall provide the following digital IO signals to each
pixel tile card:
\captionof{table}{Digital Signals (per pixel tile card)}
\begin{center}
\begin{tabular}{llllll}
Signal & Quantity & Description \\
  POSI & 4 & Bit serial transmission to pixel tile \\
  PISO & 4 & Bit serial reception from pixel tile \\
  CLK  & 1 & Clock \\
  TRIG & 1 & Trigger \\
  SYNC & 1 & Multiplexed Synchronization and Reset \\
\end{tabular}
\end{center}

Each POSI signal is received by a single ASIC, but the clock, trigger,
and sync signals are fanned out to every ASIC on the tile card.  The
PACMAN must be capable of driving the resulting capacitive load.

The nominal clock rate is $10~\rm MHz$ with a maximum of $40~\rm MHz$.
The LArPix ASIC uses an LVDS signal with custom voltage level which
the PACMAN is capable of transmitting and receiving.

\subsection{Noise}

The noise injected into the pixel tile from the controller is less than $1~\rm mV$.

\subsection{Pixel Tile Interface}

A single PACMAN card drive up to ten pixel tile cards through a single
flange card.  The interface to the eight-tile flange card is through a
6 row by 50 pin high-desnity SAMTEC connector {\tt
  SEAF-50-01-L-06-1-RA-K-TR} on the PACMAN card.

The PACMAN card shall provide a footprint for an optional 2x17
connector to drive a pixel card directly without the use of flange
card.

\subsection{Front Panel Interface}

The PACMAN card implements the front-panel interfaces listed below in
the specified quantity with specified routing.  When routing to
programmable logic (PL) is specified, the specified purpose is nominal
and can be repurposed through firmware changes.

\captionof{table}{Required Front Panel Interface}
\begin{center}
\begin{tabular}{lll}
Interface   & Quantity & Routing \\
SMA Jack    & 1 & Multiplexed to eight tile analog monitor \\
SMA Jack    & 1 & Multiplexed to eight tile adc test \\
Lemo Jack   & 2 & PL: Trigger and Sync \\
Push Button & 1 & Reset \\
LED         & 4 & PL: General Purpose
\end{tabular}
\end{center}


\subsection{Timing System Endpoint}

The PACMAN card implement a ProtoDUNE-SP timing system
endpoint\footnote{DUNE-doc-1651-v3} of a single fiber-optic link.

\subsection{Embedded Linux}

The PACMAN is an embedded Linux system and supports several standard embedded Linux interfaces:\\
\begin{tabular}{lll}
Interface & P/N & Notes \\ 
Ethernet & & Gigabit \\
SD Card & & Bootable from SD card\\
JTAG & & SoC Configuration\\
UART & & Linux Terminal\\
\end{tabular}\\  
The JTAG and UART interface are multiplexed to provide a linux terminal and firmware configuration over a single USB port. 

\appendix

\newpage
\section{PACMAN Evolution}

\captionof{table}{PACMAN Evolution}
\vskip 0.5cm
\begin{center}
\begin{tabular}{llll}
Card        & Designer      & Date & Notes \\
\hline
PACMAN v1r1 & Hillbrand     & Feb 2020 & Initial Prototype \\
PACMAN v1r2 & Karcher       & July 2020 & Single-tile capable\\
PACMAN v1r3 & Berns/Karcher & Jan 2021 & Eight-tile capable - Module 0\\
PACMAN v2 & Berns/Karcher   & Feb 2022 & LArPix-v2b, Power, TX\\
PACMAN v3 & Berns/Karcher   & TBD & Ten-tile capable \\
\end{tabular}
\end{center}

\end{document}
